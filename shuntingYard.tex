%
\documentclass{article}

\usepackage[utf8]{inputenc}
\usepackage[margin=0.5in]{geometry}
\usepackage{algorithm}
\usepackage{algpseudocode}
\usepackage{textcomp}
\usepackage{mathtools}

\begin{document}
        
        \begin{algorithm}
                \caption{in-fix to post-fix conversion}
                \label{shuntingYard}
                \begin{algorithmic}[1]
                        \Function{shuntingYard}{\textit{tokenList}[]}
                        \While {tokenList.\textsc{isEmpty}() = FALSE}
                        \State {token $ \gets $ tokenList.\textsc{pop}();}
                                \If {token.\textsc{isNum}() = TRUE}
                                        \State numStack.\textsc{push}(\textit{token.\textsc{getVal()}})
                                \ElsIf {token.\textsc{isVar}() = TRUE}
                                        \State value $ \gets $ \textsc{getVar}(\textit{token})
                                        \State numStack.\textsc{push}(\textit{value})
                                \ElsIf {token.\textsc{isLeftParen}() = TRUE}
                                        \State opStack.\textsc{push}(\textit{token.\textsc{getVal}}())
                                \ElsIf {token.\textsc{isRightParen}() = TRUE}
                                        \While {opStack.\textsc{peek}() $ \neq $ ` \texttt{(} '}
                                                \State op $ \gets $ opStack.\textsc{pop}()
                                                \State num1 $ \gets $ numStack.\textsc{pop}()
                                                \State num2 $ \gets $ numStack.\textsc{pop}()
                                                \State result $ \gets $ \textsc{eval}(\textit{op, num1, num2})
                                                \State numStack.\textsc{push}(\textit{result})
                                        \EndWhile
                                        \State opStack.\textsc{pop}()
                                \ElsIf {token.\textsc{isOperator}() = TRUE}
                                        \While {opStack.\textsc{isEmpty}() = FALSE \textbf{and} \textsc{opPrec}(\textit{opstack.}\textsc{peek()}\textit{, token}) = TRUE}
                                                \State op $ \gets $ opStack.\textsc{pop}()
                                                \State num1 $ \gets $ numStack.\textsc{pop}()
                                                \State num2 $ \gets $ numStack.\textsc{pop}()
                                                \State result $ \gets $ \textsc{eval}(\textit{op, num1, num2})
                                                \State numStack.\textsc{push(\textit{result})}
                                        \EndWhile
                                        \State opStack.\textsc{push(\textit{token})}
                                \EndIf
                        \EndWhile
                        \While {opStack.\textsc{isEmpty()} = FALSE}
                                \State op $ \gets $ opStack.\textsc{pop}()
                                \State num1 $ \gets $ numStack.\textsc{pop}()
                                \State num2 $ \gets $ numStack.\textsc{pop}()
                                \State result $ \gets $ \textsc{eval}(\textit{op, num1, num2})
                                \State numStack.\textsc{push(\textit{result})}
                        \EndWhile
                        \State \Return numStack.\textsc{pop()}
                        \EndFunction
                \end{algorithmic}
        \end{algorithm}
    
    \begin{algorithm}
    	\caption{determine operator precedence: \textsc{true} if \textit{op1} has precedence}
    	\label{opPrec}
    	\begin{algorithmic}[1]
    		\Function{opPrec}{\textit{op1, op2}}
    			\If {\textit{op2} = ` \texttt{(} ' \textbf{OR} \textit{op2} = ` \texttt{)} '}
    				\State \Return \textsc{false}
    			\ElsIf {\textbf{(}\textit{op1} = ` \texttt{/} ' \textbf{OR} \textit{op1} = ` \texttt{*} '\textbf{) AND (}\textit{op2} = ` \texttt{+} ' \textbf{OR} \textit{op2} = ` \texttt{-} '\textbf{)} }
    				\State \Return \textsc{false}
    			\Else 
    				\State \Return \textsc{true}
    			\EndIf
    		\EndFunction
    	\end{algorithmic}
    \end{algorithm}
    
    \begin{algorithm}
    	\caption{evaluate math}
    	\label{eval}
    	\begin{algorithmic}[1]
    		\Function{eval}{\textit{op, num1, num2}} 
    			\If {\textit{op} = ` \texttt{+} '}
    				\State \Return \textbf{(}\textit{num1} + \textit{num2}\textbf{)}
    			\ElsIf {\textit{op} = ` \texttt{-} '}
    				\State \Return \textbf{(}\textit{num1} - \textit{num2}\textbf{)}
    			\ElsIf {\textit{op} = ` \texttt{*} '}
    				\State \Return \textbf{(}\textit{num1} * \textit{num2}\textbf{)}
    			\ElsIf {\textit{op} = ` \texttt{/} '}
    				\If {\textit{num2} = 0}
    					\State \Return \textsc{error}
    				\Else
    					\State \Return \textbf{(}\textit{num1} / \textit{num2}\textbf{)}
    				\EndIf
    			\EndIf
    		\EndFunction
    	\end{algorithmic}
    \end{algorithm}
\end{document}
